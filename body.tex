% \logosection{\faGraduationCap}{教育经历}
\section{教育经历}
\datedline{\textbf{北京航空航天大学 \quad 北京中关村学院}}{\dateRange{2025.09}{至今}}


控制科学与工程 \quad 博士生 \quad 自动化科学与电气工程学院 \hfill 北京

\begin{itemize}
  \item 硕博连读,博士入学考试成绩优异,综合成绩排名\textcolor{ICBlue}{\textbf{3/100}},其中笔试排名\textcolor{ICBlue}{\textbf{1/100}}
  \item 入选\textcolor{ICBlue}{\textbf{国家人工智能学院(北京)即北京中关村学院}}联合培养博士生项目
  \item 研究方向: 机器人学习, 基于学习的规划与控制
  \item 预计毕业时间:2029年6月
\end{itemize}

\datedline{\textbf{北京航空航天大学}}{\dateRange{2023.09}{2025.06}}

控制科学与工程 \quad 硕士生 \quad 自动化科学与电气工程学院 \hfill {北京}
\begin{itemize}
  \item 推荐免试入学, 学业成绩排名22/83, 核心课程均分\textcolor{ICBlue}{\textbf{91.6}}
  \item 研究方向: 基于学习的规划与控制
\end{itemize}

\datedline{\textbf{东北大学}}{\dateRange{2019.09}{2023.06}}
\datedline{\tripleInfo{自动化}{工学学士}{信息科学与工程学院}}{辽宁沈阳}
\begin{itemize}
  \item 本科课程均分\textcolor{ICBlue}{\textbf{91.2}}, 综合排名\textcolor{ICBlue}{\textbf{10/235}}, 毕业设计获得优秀
  \item 大一下以工程力学专业第一的成绩转专业至自动化
\end{itemize}

% \logosection{\faWrench}{项目经历}
\section{科研经历}
\textbf{已发表论文:}
\begin{itemize}
  \item \textcolor{ICBlue}{\textbf{Z. Shen}} and Q. Quan, “DOPT: D-learning with Off-Policy Target toward Sample Efficiency and Fast Convergence Control,” in \textcolor{ICBlue}{\textbf{2025 IEEE International Conference on Robotics and Automation (ICRA)}}, 2025, pp. 9637–9643.
  \item J. Liu, C. Wang, \textcolor{ICBlue}{\textbf{Z. Shen}} and Q. Quan, “DL-Clip: Online D-Learning with Clipping Operation for Fast Model-Free Stabilizing Control,” in \textcolor{ICBlue}{\textbf{2025 IEEE/RSJ International Conference on Intelligent Robots and Systems (IROS)}}, 2025, accepted.
  \item J. Liu, S. Chen and \textcolor{ICBlue}{\textbf{Z. Shen}}, "Analysis of Metro Traveling Crowd Based on Kernel K-means Clustering Algorithm-Take Shenzhen Metro as an example," \textcolor{ICBlue}{\textbf{2021 2nd International Conference on Computer Science and Management Technology (ICCSMT)}}, Shanghai, China, 2021, pp. 271-278.
\end{itemize}

\textbf{已/待投递论文:}
\begin{itemize}
  \item H. Cao, \textcolor{ICBlue}{\textbf{Z. Shen}} and Q. Quan, “Learning to Adapt: Reptile-D-learning for Robust and Efficient Control Under Parametric Uncertainty,” submitted to \textcolor{ICBlue}{\textbf{ICRA 2026}}.
  \item P. Mao, S. Lv, C. Min, \textcolor{ICBlue}{\textbf{Z. Shen}} and Q. Quan, “An Efficient Real-Time Planning Method for Swarm Robotics Based on an Optimal Virtual Tube,” submitted to \textcolor{ICBlue}{\textbf{T-RO}}.
\end{itemize}

\datedline{\textbf{空中具身智能}}{\dateRange{2025.10}{至今}}
\datedline{\biInfo{课题组科研项目}{负责人}}{北京}
\begin{itemize}
  \item 带领课题组同学进行空中具身智能研究。
  \item 内容包括空中具身智能平台调研与设计, 应用场景设计, 确定技术路线等前期工作。
\end{itemize}

\datedline{\textbf{基于D学习的多旋翼控制与sim2real迁移}}{\dateRange{2025.03}{至今}}
\datedline{\biInfo{课题组科研项目}{负责人}}{北京}
\begin{itemize}
  \item 面向多旋翼飞行器的稳定与跟踪控制问题, 提出了引入了先验信息的分层D学习方法。
  \item 面向高维复杂非线性系统对D学习理论进行改进, 基于IsaacLab搭建针对不同任务的并行化训练环境, 分层D学习控制器的设计与实现。
  \item 在D学习中引入先验动力学与Lyapunov函数信息, 加快D学习的训练收敛速度。
  \item Sim2Real 的场景搭建, 包括多旋翼飞行器, 动捕平台, 实验环境等。
\end{itemize}

\datedline{\textbf{无人机集群障碍穿越实验}}{\dateRange{2024.12}{2025.02}}
\datedline{\biInfo{课题组科研项目}{核心成员}}{宁波}
\begin{itemize}
  \item 参与多无人机集群障碍穿越实验, 包括前期设备调试, 实验环境搭建, 算法实机部署, 实验数据采集。
  \item 参与结题报告撰写与修订。
\end{itemize}

\datedline{\textbf{面向动力学参数不确定的Meta D学习}}{\dateRange{2024.09}{2025.08}}
\datedline{\biInfo{课题组科研项目}{核心成员}}{北京}
\begin{itemize}
  \item 提出一种面向动力学模型参数不确定性的 Meta D-learning 的控制方法 实现了对参数不确定性系统的稳定控制。
  \item 项目中主要贡献了核心算法代码与测试平台。
  \item 成果提交于\textcolor{ICBlue}{\textbf{2026 IEEE International Conference on Robotics \& Automation (ICRA) 第二作者}}。
\end{itemize}


\datedline{\textbf{带有Cilp操作的稳定D学习}}{\dateRange{2024.05}{2025.03}}
\datedline{\biInfo{课题组科研项目}{核心成员}}{北京}
\begin{itemize}
  \item 提出一种 D-learning 的改进方法, 引入了裁剪操作, 与强化学习 (PPO) 相比实现了更快更稳定的训练, 并应用于视觉伺服 (IBVS)任务。
  \item 项目中主要贡献了核心算法代码, 并协助在动捕环境下使用 Tello EDU 平台开展 IBVS 实验。
  \item 成果接收于\textcolor{ICBlue}{\textbf{2025 IEEE/RSJ International Conference on Intelligent Robots \& Systems (IROS) 第三作者}}, 并在大会进行口头汇报。
\end{itemize}


\datedline{\textbf{离策略高效D学习}}{\dateRange{2024.05}{2024.09}}
\datedline{\biInfo{课题组科研项目}{负责人}}{北京}
\begin{itemize}
  \item 提出一种 Off-Policy Model-free D-learning 方法 (DOPT), 首次实现了 \href{https://proceedings.mlr.press/v270/quan25a.html}{D-learning} 和 DOPT 的在线迭代, 并应用于机器人控制任务。
  \item 与强化学习 (DDPG) 相比, DOPT 实现了更高的样本利用效率, 训练出的策略在机器人控制任务中具有 Lyapunov 稳定性保证, 更强的鲁棒性, 更快收敛速度, 以及更小的稳态误差。
  \item 成果接收于\textcolor{ICBlue}{\textbf{2025 IEEE International Conference on Robotics \& Automation (ICRA) 第一作者}}, 并在大会进行口头汇报。
\end{itemize}

\datedline{\textbf{四旋翼飞行器以及飞控的开发}}{\dateRange{2022.01}{2022.06}}
\datedline{\biInfo{个人兴趣}{负责人}}{辽宁沈阳}
\begin{itemize}
  \item 从零开始四旋翼飞行器平台的搭建, 包括硬件选型, 电路设计, 软件编程等。
  \item 基于 Atmega328p 进行飞控软硬件的设计, mpu6050 运动处理传感器的应用与数据处理, 飞行器的姿态算解等。
  \item 申请实用新型专利, 一种变轴距二次折叠式无人机架。
\end{itemize}

\datedline{\textbf{人工肌肉的随机应征控制}}{\dateRange{2021.08}{2022.06}}
\datedline{\biInfo{课题组科研项目}{负责人}}{辽宁沈阳}
\begin{itemize}
  \item 研究了人工肌肉的随机应征控制问题, 提出了一种基于随机应征模型的控制方法。
  \item 分析存在失效元胞的系统状态, 进行失效元胞的随机应征控制系统的建模与仿真,随机应征控制系统的动态性能优化。
  \item 内容总结、论文的撰写与投稿。
\end{itemize}

\datedline{\textbf{针对深圳地铁搭乘数据基于核函数的k-means聚类分析}}{\dateRange{2021.03}{2021.10}}
\datedline{\biInfo{学生科研实践}{核心成员}}{辽宁沈阳}
\begin{itemize}
  \item 针对深圳地铁搭乘数据, 采用基于核函数的 k-means 聚类算法进行分析, 分析了地铁搭乘数据的聚类结果, 发现了不同用户的搭乘模式。
  \item 成果接收于\textbf{2021 IEEE International Conference on Computer Science and Management Technology (ICCSMT) 共同第一作者}。
\end{itemize}

% \section{竞赛经历}

% \datedline{\textbf{2023华为杯数学建模竞赛}}{\dateRange{2023.09}{2023.10}}
% \datedline{\biInfo{学生竞赛}{核心成员}}{北京}
% \begin{itemize}
%   \item 三等奖
% \end{itemize}

% \datedline{\textbf{2022美国大学生数学建模比赛}}{\dateRange{2022.01}{2022.02}}
% \datedline{\biInfo{学生竞赛}{负责人}}{辽宁沈阳}
% \begin{itemize}
%   \item Honorable Mention
% \end{itemize}

% \datedline{\textbf{2021全国大学生数学建模竞赛}}{\dateRange{2021.08}{2021.09}}
% \datedline{\biInfo{学生竞赛}{负责人}}{辽宁沈阳}
% \begin{itemize}
%   \item 辽宁省三等奖
% \end{itemize}

% \datedline{\textbf{2021东北大学电子设计大赛}}{\dateRange{2021.06}{2021.07}}
% \datedline{\biInfo{学生竞赛}{核心成员}}{辽宁沈阳}
% \begin{itemize}
%   \item 二等奖
% \end{itemize}

\section{获奖情况}
\datedline{北京航空航天大学一等学业奖学金}{2024}
\datedline{北京航空航天大学校级优秀团员}{2024, 2025}
\datedline{北京航空航天大学二等新生奖学金}{2023}
\datedline{东北大学优秀学生}{2023, 2022, 2020}
\datedline{东北大学二等奖学金}{2023, 2022, 2020}
\datedline{东北大学三等奖学金}{2021}
\datedline{东北大学优秀学业特长个人}{2021}
\datedline{大连校友会二等奖学金}{2020}

\section{技能}
# TODO 需要具体化一些技能
\begin{itemize}[parsep=0.5ex]
  \item 编程语言: Python = Matlab = Latex > C++ > Shell
  \item 工具: PyTorch, IsaacLab, ROS, Git
  \item 语言: 英语 - 熟练 ( IELTS 7.5 / CET-6 587 )
\end{itemize}

\section{其他}
\begin{itemize}[parsep=0.5ex]
  \item 技术博客: \href{https://blog.csdn.net/longger\_r}{https://blog.csdn.net/longger\_r}
  \item GitHub: \href{https://github.com/Shenzhaolong1330}{https://github.com/Shenzhaolong1330}
  \item 个人主页: \href{https://shenzhaolong1330.github.io}{https://shenzhaolong1330.github.io}
\end{itemize}

%%%% 如果多页简历,可以手动在适当位置插入 \newpage 或者 \clearpage 开始新一页
